 In this example, we navigate a simple X\+M\+L file, and read some interesting text. Note that this example doesn't use error checking; working code should check for null pointers when walking an X\+M\+L tree, or use X\+M\+L\+Handle.

(The X\+M\+L is an excerpt from \char`\"{}dream.\+xml\char`\"{}).


\begin{DoxyCodeInclude}

\end{DoxyCodeInclude}


The structure of the X\+M\+L file is\+:


\begin{DoxyItemize}
\item (declaration) 
\item (dtd stuff) 
\item Element \char`\"{}\+P\+L\+A\+Y\char`\"{} 
\begin{DoxyItemize}
\item Element \char`\"{}\+T\+I\+T\+L\+E\char`\"{} 
\begin{DoxyItemize}
\item Text \char`\"{}\+A Midsummer Night's Dream\char`\"{} 
\end{DoxyItemize}
\end{DoxyItemize}
\end{DoxyItemize}

For this example, we want to print out the title of the play. The text of the title (what we want) is child of the \char`\"{}\+T\+I\+T\+L\+E\char`\"{} element which is a child of the \char`\"{}\+P\+L\+A\+Y\char`\"{} element.

We want to skip the declaration and dtd, so the method First\+Child\+Element() is a good choice. The First\+Child\+Element() of the Document is the \char`\"{}\+P\+L\+A\+Y\char`\"{} Element, the First\+Child\+Element() of the \char`\"{}\+P\+L\+A\+Y\char`\"{} Element is the \char`\"{}\+T\+I\+T\+L\+E\char`\"{} Element.


\begin{DoxyCodeInclude}

\end{DoxyCodeInclude}


We can then use the convenience function Get\+Text() to get the title of the play.


\begin{DoxyCodeInclude}

\end{DoxyCodeInclude}


Text is just another Node in the X\+M\+L D\+O\+M. And in fact you should be a little cautious with it, as text nodes can contain elements.

\begin{DoxyVerb}Consider: A Midsummer Night's <b>Dream</b>
\end{DoxyVerb}


It is more correct to actually query the Text Node if in doubt\+:


\begin{DoxyCodeInclude}

\end{DoxyCodeInclude}


Noting that here we use First\+Child() since we are looking for X\+M\+L\+Text, not an element, and To\+Text() is a cast from a Node to a X\+M\+L\+Text. 