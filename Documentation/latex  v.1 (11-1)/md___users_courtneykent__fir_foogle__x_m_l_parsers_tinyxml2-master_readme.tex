

Tiny\+X\+M\+L-\/2 is a simple, small, efficient, C++ X\+M\+L parser that can be easily integrated into other programs.

The master is hosted on github\+: \href{https://github.com/leethomason/tinyxml2}{\tt https\+://github.\+com/leethomason/tinyxml2}

The online H\+T\+M\+L version of these docs\+: \href{http://grinninglizard.com/tinyxml2docs/index.html}{\tt http\+://grinninglizard.\+com/tinyxml2docs/index.\+html}

Examples are in the \char`\"{}related pages\char`\"{} tab of the H\+T\+M\+L docs.

\subsection*{What it does. }

In brief, Tiny\+X\+M\+L-\/2 parses an X\+M\+L document, and builds from that a Document Object Model (D\+O\+M) that can be read, modified, and saved.

X\+M\+L stands for \char`\"{}e\+Xtensible Markup Language.\char`\"{} It is a general purpose human and machine readable markup language to describe arbitrary data. All those random file formats created to store application data can all be replaced with X\+M\+L. One parser for everything.

\href{http://en.wikipedia.org/wiki/XML}{\tt http\+://en.\+wikipedia.\+org/wiki/\+X\+M\+L}

There are different ways to access and interact with X\+M\+L data. Tiny\+X\+M\+L-\/2 uses a Document Object Model (D\+O\+M), meaning the X\+M\+L data is parsed into a C++ objects that can be browsed and manipulated, and then written to disk or another output stream. You can also construct an X\+M\+L document from scratch with C++ objects and write this to disk or another output stream. You can even use Tiny\+X\+M\+L-\/2 to stream X\+M\+L programmatically from code without creating a document first.

Tiny\+X\+M\+L-\/2 is designed to be easy and fast to learn. It is one header and one cpp file. Simply add these to your project and off you go. There is an example file -\/ xmltest.\+cpp -\/ to get you started.

Tiny\+X\+M\+L-\/2 is released under the Z\+Lib license, so you can use it in open source or commercial code. The details of the license are at the top of every source file.

Tiny\+X\+M\+L-\/2 attempts to be a flexible parser, but with truly correct and compliant X\+M\+L output. Tiny\+X\+M\+L-\/2 should compile on any reasonably C++ compliant system. It does not rely on exceptions, R\+T\+T\+I, or the S\+T\+L.

\subsection*{What it doesn't do. }

Tiny\+X\+M\+L-\/2 doesn't parse or use D\+T\+Ds (Document Type Definitions) or X\+S\+Ls (e\+Xtensible Stylesheet Language.) There are other parsers out there that are much more fully featured. But they are also much bigger, take longer to set up in your project, have a higher learning curve, and often have a more restrictive license. If you are working with browsers or have more complete X\+M\+L needs, Tiny\+X\+M\+L-\/2 is not the parser for you.

\subsection*{Tiny\+X\+M\+L-\/1 vs. Tiny\+X\+M\+L-\/2 }

Tiny\+X\+M\+L-\/2 is now the focus of all development, well tested, and your best choice unless you have a requirement to maintain Tiny\+X\+M\+L-\/1 code.

Tiny\+X\+M\+L-\/2 uses a similar A\+P\+I to Tiny\+X\+M\+L-\/1 and the same rich test cases. But the implementation of the parser is completely re-\/written to make it more appropriate for use in a game. It uses less memory, is faster, and uses far fewer memory allocations.

Tiny\+X\+M\+L-\/2 has no requirement for S\+T\+L, but has also dropped all S\+T\+L support. All strings are query and set as 'const char$\ast$'. This allows the use of internal allocators, and keeps the code much simpler.

Both parsers\+:


\begin{DoxyEnumerate}
\item Simple to use with similar A\+P\+Is.
\item D\+O\+M based parser.
\item U\+T\+F-\/8 Unicode support. \href{http://en.wikipedia.org/wiki/UTF-8}{\tt http\+://en.\+wikipedia.\+org/wiki/\+U\+T\+F-\/8}
\end{DoxyEnumerate}

Advantages of Tiny\+X\+M\+L-\/2


\begin{DoxyEnumerate}
\item The focus of all future dev.
\item Many fewer memory allocation (1/10th to 1/100th), uses less memory (about 40\% of Tiny\+X\+M\+L-\/1), and faster.
\item No S\+T\+L requirement.
\item More modern C++, including a proper namespace.
\item Proper and useful handling of whitespace
\end{DoxyEnumerate}

Advantages of Tiny\+X\+M\+L-\/1


\begin{DoxyEnumerate}
\item Can report the location of parsing errors.
\item Support for some C++ S\+T\+L conventions\+: streams and strings
\item Very mature and well debugged code base.
\end{DoxyEnumerate}

\subsection*{Features }

\subsubsection*{Memory Model}

An X\+M\+L\+Document is a C++ object like any other, that can be on the stack, or new'd and deleted on the heap.

However, any sub-\/node of the Document, X\+M\+L\+Element, X\+M\+L\+Text, etc, can only be created by calling the appropriate X\+M\+L\+Document\+::\+New\+Element, New\+Text, etc. method. Although you have pointers to these objects, they are still owned by the Document. When the Document is deleted, so are all the nodes it contains.

\subsubsection*{White Space}

\paragraph*{Whitespace Preservation (default)}

Microsoft has an excellent article on white space\+: \href{http://msdn.microsoft.com/en-us/library/ms256097.aspx}{\tt http\+://msdn.\+microsoft.\+com/en-\/us/library/ms256097.\+aspx}

By default, Tiny\+X\+M\+L-\/2 preserves white space in a (hopefully) sane way that is almost complient with the spec. (Tiny\+X\+M\+L-\/1 used a completely different model, much more similar to 'collapse', below.)

As a first step, all newlines / carriage-\/returns / line-\/feeds are normalized to a line-\/feed character, as required by the X\+M\+L spec.

White space in text is preserved. For example\+: \begin{DoxyVerb}<element> Hello,  World</element>
\end{DoxyVerb}


The leading space before the \char`\"{}\+Hello\char`\"{} and the double space after the comma are preserved. Line-\/feeds are preserved, as in this example\+: \begin{DoxyVerb}<element> Hello again,  
          World</element>
\end{DoxyVerb}


However, white space between elements is {\bfseries not} preserved. Although not strictly compliant, tracking and reporting inter-\/element space is awkward, and not normally valuable. Tiny\+X\+M\+L-\/2 sees these as the same X\+M\+L\+: \begin{DoxyVerb}<document>
    <data>1</data>
    <data>2</data>
    <data>3</data>
</document>

<document><data>1</data><data>2</data><data>3</data></document>
\end{DoxyVerb}


\paragraph*{Whitespace Collapse}

For some applications, it is preferable to collapse whitespace. Collapsing whitespace gives you \char`\"{}\+H\+T\+M\+L-\/like\char`\"{} behavior, which is sometimes more suitable for hand typed documents.

Tiny\+X\+M\+L-\/2 supports this with the 'whitespace' parameter to the X\+M\+L\+Document constructor. (The default is to preserve whitespace, as described above.)

However, you may also use C\+O\+L\+L\+A\+P\+S\+E\+\_\+\+W\+H\+I\+T\+E\+S\+P\+A\+C\+E, which will\+:


\begin{DoxyItemize}
\item Remove leading and trailing whitespace
\item Convert newlines and line-\/feeds into a space character
\item Collapse a run of any number of space characters into a single space character
\end{DoxyItemize}

Note that (currently) there is a performance impact for using C\+O\+L\+L\+A\+P\+S\+E\+\_\+\+W\+H\+I\+T\+E\+S\+P\+A\+C\+E. It essentially causes the X\+M\+L to be parsed twice.

\subsubsection*{Entities}

Tiny\+X\+M\+L-\/2 recognizes the pre-\/defined \char`\"{}character entities\char`\"{}, meaning special characters. Namely\+: \begin{DoxyVerb}&amp;   &
&lt;    <
&gt;    >
&quot;  "
&apos;  '
\end{DoxyVerb}


These are recognized when the X\+M\+L document is read, and translated to their U\+T\+F-\/8 equivalents. For instance, text with the X\+M\+L of\+: \begin{DoxyVerb}Far &amp; Away
\end{DoxyVerb}


will have the Value() of \char`\"{}\+Far \& Away\char`\"{} when queried from the X\+M\+L\+Text object, and will be written back to the X\+M\+L stream/file as an ampersand.

Additionally, any character can be specified by its Unicode code point\+: The syntax \char`\"{}\&\#x\+A0;\char`\"{} or \char`\"{}\&\#160;\char`\"{} are both to the non-\/breaking space characher. This is called a 'numeric character reference'. Any numeric character reference that isn't one of the special entities above, will be read, but written as a regular code point. The output is correct, but the entity syntax isn't preserved.

\subsubsection*{Printing}

\paragraph*{Print to file}

You can directly use the convenience function\+: \begin{DoxyVerb}XMLDocument doc;
...
doc.SaveFile( "foo.xml" );
\end{DoxyVerb}


Or the X\+M\+L\+Printer class\+: \begin{DoxyVerb}XMLPrinter printer( fp );
doc.Print( &printer );
\end{DoxyVerb}


\paragraph*{Print to memory}

Printing to memory is supported by the X\+M\+L\+Printer. \begin{DoxyVerb}XMLPrinter printer;
doc.Print( &printer );
// printer.CStr() has a const char* to the XML
\end{DoxyVerb}


\paragraph*{Print without an X\+M\+L\+Document}

When loading, an X\+M\+L parser is very useful. However, sometimes when saving, it just gets in the way. The code is often set up for streaming, and constructing the D\+O\+M is just overhead.

The Printer supports the streaming case. The following code prints out a trivially simple X\+M\+L file without ever creating an X\+M\+L document. \begin{DoxyVerb}XMLPrinter printer( fp );
printer.OpenElement( "foo" );
printer.PushAttribute( "foo", "bar" );
printer.CloseElement();
\end{DoxyVerb}


\subsection*{Examples }

\paragraph*{Load and parse an X\+M\+L file.}

\begin{DoxyVerb}/* ------ Example 1: Load and parse an XML file. ---- */    
{
    XMLDocument doc;
    doc.LoadFile( "dream.xml" );
}
\end{DoxyVerb}


\paragraph*{Lookup information.}

\begin{DoxyVerb}/* ------ Example 2: Lookup information. ---- */    
{
    XMLDocument doc;
    doc.LoadFile( "dream.xml" );

    // Structure of the XML file:
    // - Element "PLAY"      the root Element, which is the 
    //                       FirstChildElement of the Document
    // - - Element "TITLE"   child of the root PLAY Element
    // - - - Text            child of the TITLE Element

    // Navigate to the title, using the convenience function,
    // with a dangerous lack of error checking.
    const char* title = doc.FirstChildElement( "PLAY" )->FirstChildElement( "TITLE" )->GetText();
    printf( "Name of play (1): %s\n", title );

    // Text is just another Node to TinyXML-2. The more
    // general way to get to the XMLText:
    XMLText* textNode = doc.FirstChildElement( "PLAY" )->FirstChildElement( "TITLE" )->FirstChild()->ToText();
    title = textNode->Value();
    printf( "Name of play (2): %s\n", title );
}
\end{DoxyVerb}


\subsection*{Using and Installing }

There are 2 files in Tiny\+X\+M\+L-\/2\+:
\begin{DoxyItemize}
\item tinyxml2.\+cpp
\item \hyperlink{tinyxml2_8h_source}{tinyxml2.\+h}
\end{DoxyItemize}

And additionally a test file\+:
\begin{DoxyItemize}
\item xmltest.\+cpp
\end{DoxyItemize}

Simply compile and run. There is a visual studio 2010 project included, a simple Makefile, an X\+Code project, a Code\+::\+Blocks project, and a cmake C\+Make\+Lists.\+txt included to help you. The top of tinyxml.\+h even has a simple g++ command line if you are are $\ast$nix and don't want to use a build system.

\subsection*{Versioning }

Tiny\+X\+M\+L-\/2 uses semantic versioning. \href{http://semver.org/}{\tt http\+://semver.\+org/} Releases are now tagged in github.

Note that the major version will (probably) change fairly rapidly. A\+P\+I changes are fairly common.

\subsection*{Documentation }

The documentation is build with Doxygen, using the 'dox' configuration file.

\subsection*{License }

Tiny\+X\+M\+L-\/2 is released under the zlib license\+:

This software is provided 'as-\/is', without any express or implied warranty. In no event will the authors be held liable for any damages arising from the use of this software.

Permission is granted to anyone to use this software for any purpose, including commercial applications, and to alter it and redistribute it freely, subject to the following restrictions\+:


\begin{DoxyEnumerate}
\item The origin of this software must not be misrepresented; you must not claim that you wrote the original software. If you use this software in a product, an acknowledgment in the product documentation would be appreciated but is not required.
\item Altered source versions must be plainly marked as such, and must not be misrepresented as being the original software.
\item This notice may not be removed or altered from any source distribution.
\end{DoxyEnumerate}

\subsection*{Contributors }

Thanks very much to everyone who sends suggestions, bugs, ideas, and encouragement. It all helps, and makes this project fun.

The original Tiny\+X\+M\+L-\/1 has many contributors, who all deserve thanks in shaping what is a very successful library. Extra thanks to Yves Berquin and Andrew Ellerton who were key contributors.

Tiny\+X\+M\+L-\/2 grew from that effort. Lee Thomason is the original author of Tiny\+X\+M\+L-\/2 (and Tiny\+X\+M\+L-\/1) but Tiny\+X\+M\+L-\/2 has been and is being improved by many contributors.

Thanks to John Mackay at \href{http://john.mackay.rosalilastudio.com}{\tt http\+://john.\+mackay.\+rosalilastudio.\+com} for the Tiny\+X\+M\+L-\/2 logo! 